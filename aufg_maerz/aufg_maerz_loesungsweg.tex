\documentclass[10pt]{scrartcl}
\usepackage[latin1]{inputenc}
\usepackage{ngerman}
\usepackage[babel]{csquotes}
\author{Wolfgang Keller}
\title{L�sung zur Aufgabe des Monats M�rz}
\begin{document}
\maketitle
\section{L�sung zur Aufgabe des Monats M�rz}
Die G�ltigkeit der Aussage f�r n=1, 2 St�dte ist offensichtlich, da es f�r eine Stadt keine anderen zu erreichenden St�dte gibt und damit die einzige vorhandene Stadt trivialerweise Zentralstadt ist und f�r $n=2$ muss gibt es nur den Fall, dass von einer der beiden St�dte (o. E. nennen wir diese A) die andere (welche wir o. E. B nennen wollen) erreichbar ist. Somit ist offensichtlich A Zentralstadt.

Nun zum Induktionsschritt.

Gelte die Aussage f�r m St�dte mit $m\geq 2$. Also gibt es eine Stadt Z (Zentralstadt), so dass jede der ersten m (au�er Z) St�dte von Z entweder direkt oder �ber h�chstens eine Zwischenstadt erreichbar ist.

Die m+1-ste Stadt wollen wir X nennen.

\paragraph{Fall 1:} X ist von Z direkt erreichbar. In diesem Fall ist Z auch Zentralstadt der m+1 St�dte und die Aussage gilt damit auch f�r m+1 St�dte.

\paragraph{Fall 2:} X ist von Z nicht direkt erreichbar, aber es gibt eine von Z direkt erreichbare Stadt, von der X direkt erreichbar ist. Auch in diesem Fall ist Z auch Zentralstadt der m+1 St�dte und die Aussage gilt damit auch f�r m+1 St�dte.

\paragraph{Fall 3:} X ist von Z nicht direkt erreichbar und es gibt keine von Z direkt erreichbare Stadt, von der die Stadt X direkt erreichbar ist.

Jede von Z direkt erreichbare Stadt A ist demnach auch von X direkt erreichbar (da entweder eine Stra�e von A nach X oder X nach A existiert. Da nach Fallvoraussetzung keine Stra�e von A nach X existiert (sonst w�re X von Z �ber die Zwischenstadt A erreichbar), muss demnach eine Stra�e von X nach A existieren).

Z muss sowieso von X direkt erreichbar sein, sonst w�re mit dem selben Argument X von Z aus direkt erreichbar, was im Widerspruch zur Voraussetzung des Falls st�nde. 

Also sind alle von Z direkt erreichbaren St�dte (sowie Z) von X direkt erreichbar. Damit sind alle von Z �ber eine Zwischenstadt erreichbaren St�dte auch von X �ber eine Zwischenstadt erreichbar. Somit ist X Zentralstadt der m+1 St�dte und die Aussage gilt demnach auch f�r m+1 St�dte.
\end{document}