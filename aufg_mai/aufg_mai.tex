\documentclass[10pt]{scrartcl}
\usepackage[latin1]{inputenc}
\usepackage[ngerman]{babel}
\author{Wolfgang Keller}
\title{Gro�e Primzahlen -- Aufgabe des Monats Mai}
\begin{document}
\maketitle
\section{Aufgabe}
Die Zahl Googol (aus der sich �brigens der Name der Suchmaschine "`Google"' ableitet) ist definiert als $10^{100}$.

Man zeige: es gibt zwei Primzahlen p und q, die sich um mindestens Googol unterscheiden und zwischen denen sich keine weitere Primzahl befindet.

\section{Bonusaufgabe}
Man sch�tze die Gr��enordnungen von p und q ab (im Sinne der Anzahl der Dezimalstellen). Die folgenden S�tze und Ungleichungen k�nnten hierbei hilfreich sein:

\subsection{Die Stirling-Ungleichung}
Mit $n!:=n (n-1) \ldots 1$ (wobei $0!:=1$ gilt, wie es allgemein �blich bei der Definition eines leeren Produkts ist) gilt:

\begin{displaymath}
	\sqrt{2\pi} n^{n+\frac{1}{2}} e^{-n+\frac{1}{12 n+1}} < n!< \sqrt{2\pi} n^{n+\frac{1}{2}} e^{-n+\frac{1}{12 n}}
\end{displaymath}

\subsection{Satz �ber Primzahldichte}

F�r jede nat�rliche Zahl n gilt, dass es eine Primzahl p mit $n\leq p\leq 2 n$ gibt.
\end{document}